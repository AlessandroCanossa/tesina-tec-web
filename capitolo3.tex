\section{Tecnologie usate}

\subsection{Backend}
Per il backend della web app è stato utilizzato il framework web visto a lezione, ovvero Django.
Si è scelto di adottare questo framework perché permette una grande velocità
di sviluppo una volta che si è capito il meccanismo di funzionamento. Permette
anche di astrarre tutte le complessità legate alla gestione di un database e alla
creazione dinamica dei templates delle pagine con i dati ottenuti dal database.
\\Per la maggior parte delle viste è stato utilizzato il template engine di Django.
Ogniuna di tali viste è legata ad un template e gestisce la logica di ottenimento di quello che
deve essere fatto vedere a schermo e/o la modifica di questi oggetti all’interno
del database.
\\Sono state però create anche delle viste che forniscono dei più semplici servizi,
come per esempio l'aggiunta di un voto ad un comic, le quali restitusicono solo
messaggi di successo o di errore.
\\Come database è stato utilizzato \textbf{SQLite3}, siccome è molto leggero,
l'app non necessita di funzioni di gestione del database particolari e anche perché
è il database che viene fornito di base da Django.


\subsection{Frontend}
Per la realizzazione del Frontend sono stati usati i template di Django,
che permettono di creare pagine dinamiche con i dati ottenuti dal database.
Inoltre forniscono anche la possibilità di creare componenti HTML riutilizzabili
e la possibilità di estendere pagine con altri template mediante l'utilizzo dei blocchi.
\\Come libreria grafica è stata utilizzata \textbf{Bootstrap}, per la facilità di
creazione degli oggetti.
\\Per la dinamicità delle pagine è stato usato anche \textbf{Javascript}, e in particolare
\textbf{AJAX}, per la gestione di bottoni e piccoli form, ad esempio per il tasto di voto.