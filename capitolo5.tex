\section{Scelte effetuate}

\subsection{Viste}
Le viste sono state realizzate mediante funzioni, invece che classi,
per la possibilità di personalizzazione di fuzionamento,
a discapito del possibile utilizzo di mixins.
Questo ha portato a dover effettuare spesso controlli sui permessi degli utenti.

\subsection{AJAX}
L'utilizzo di AJAX è stato scelto per la possibilità di gestione dinamica
di pagine altrimenti statiche. Principalmente è stato utilizzato per bottoni
e piccoli form che non avrebbe avuto senso gestire con pagine separate.

\subsection{AUTH}
L'utilizzo di django auth è stato limitato al solo login e logout,
per evitare di andare a creare pagine superflue, come quella di conferma cambio password.
\\Per le notifiche di conferma cambio password e altre notifiche simili sono stati utilizzati
gli \textit{alert} di Javascript in combo con AJAX.

\subsection{Pannello admin}
Il pannello di gestione dell'admin è stato esteso con funzioni di ricerca e filtraggio per ogni modello presente.
Questo permette una ricerca più efficiente e una visualizzazione più compatta.

\subsection{Immagini}
La gestione delle immagini inizialmente si è rivelata abbastanza complicata,
ma grazie al pacchetto \textit{django-dynamic-image} si è potuto creare dinamicamente
i path di salvataggio delle immagini, in base a comic e numero di capitolo.